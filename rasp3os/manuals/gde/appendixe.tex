\documentstyle[a4,11pt]{report}
%
\newenvironment{display}%
   {\begin{center}
    \begin{minipage}[t]{5.4in}}%
   {\end{minipage}
    \end{center}}
%
\newenvironment{ex}%
   {\begin{display}}%
   {\end{display}}
%
\begin{document} 
\appendix
\setcounter{chapter}{4}
\chapter{Grammar Development Environment version 1.35: New Facilities}

The fourth release of the Alvey Natural Language Tools (ANLT) contains
version 1.35 of the Grammar Development Environment (GDE); this
version is substantially the same as that
described in the main GDE documentation on this tape (and also in
Cambridge University Computer Laboratory Technical Report no.\ 233). 
Version 1.35 contains a couple of additional facilities in the area of
defining semantic formulae in ID rules and metarules, a few user interface
improvements, and uses a different internal representation for the object
grammar. This document describes the differences.


\section*{Semantics of Kleene-repeated Categories}

Kleene-repeated categories in ID and PS rules are categories which may
be instantiated an indefinite number of times at parse time. The standard
type of semantic formulae for ID rules (referring to the semantics of
daughters by {\tt 1}, {\tt 2} etc.) makes little sense when one of the daughters is
repeated. What is normally required for such rules is the ability to splice
a whole sequence of formulae stemming from a repeated daughter into the result
formula. In version 1.35, this can be done by using the notation
{\tt 1+}, {\tt 2+} etc.; {\tt 1+} means splice in the semantics of
all daughter nodes in
the parse tree starting from the node corresponding to the first occurrence
of the left-most rule daughter and continuing to the right; {\tt 2+} means start
from the first occurrence of the second rule daughter, and so on.
ID rules introducing {\tt VP} and {\tt N} co-ordination can thus be written as:
\begin{ex}
\begin{verbatim}
VP/COORD : VP[COORD +] --> VP[CONJ EITHER], ( VP[CONJ OR] )+ :
   (lambda (x) (OR (1 x) (2+ x))).

N/COORD : N[COORD +] --> ( N[CONJ NULL] )+, ( N[CONJ AND] )+ :
   (AND 1+).
\end{verbatim}
\end{ex}
producing for the verb phrase {\it either smiles or laughs or cries}, the semantics
\begin{ex}
\begin{verbatim}
(lambda (x) (OR (SMILE x) (LAUGH x) (CRY x)))
\end{verbatim}
\end{ex}
and for {\it man woman and child}, the semantics
\begin{ex}
\begin{verbatim}
(AND MAN WOMAN CHILD)
\end{verbatim}
\end{ex}
When a repeated index is applied to one or more arguments (as in the
{\tt (2+ x)} in the {\tt VP} rule above), each formula is first applied to the
arguments before being spliced in.

Alternative clauses containing conditions can be associated with kleene daughters,
and when that daughter or any subsequent one satisfies a condition, the part of
the output semantics referring to the daughter is spliced into the final
result. E.g. with the rule
\begin{ex}
\begin{verbatim}
N2/COORD : N2[COORD +] --> ( N2[CONJ NULL] )+, ( N2[CONJ AND] )+ : 
   1 = [PLU -], (lambda (x) (and (1+ (SING x)))) : 
   1 = [PLU +], (lambda (x) (and (1+ (PLU x)))). 
\end{verbatim}
\end{ex}
and the local tree
\begin{ex}
\begin{verbatim}
                     N2[COORD +]
               /         |         \
           N2[PLU -]  N2[PLU -]  N2[PLU +]
\end{verbatim}
\end{ex}
(which might result from parsing {\it kim, sandy and the students}) the
semantics of the mother node would be
\begin{ex}
\begin{verbatim}
(lambda (x) (and (1 (SING x)) (2 (SING x)) (3 (PLU x))))
\end{verbatim}
\end{ex}


\section*{Metarule Semantic Conditions}

Version 1.35 allows syntactic conditions to be associated with
metarule semantic formulae (in 1.32 conditions were only allowed in ID and PS
rule semantics). Metarule conditions are stated in a similar way to ones in ID
rules, and they are used to constrain which metarule formulae apply to which ID
rule formulae, or add new conditions to the output expanded rule. The metarule
condition's category index controls which of these two roles a particular condition
plays. The categories in a metarule are numbered from zero, starting with the LHS
mother, continuing with the LHS daughters, then the RHS mother and finally
the RHS daughters.
A condition whose index refers to a category on the metarule LHS allows
its semantic formula to apply only to input ID rule formulae which either have a 
matching condition on the same daughter, or which have no condition on that daughter.
A condition whose index refers to a category on the metarule RHS is
simply added as a condition on the semantic formula on the output rule. Input
ID rule conditions which are not matched by any metarule condition are
carried through unchanged to the output rule.

As an example, the metarule {\tt PASS} below has three semantic formula,
each of which applies only to input ID rule formulae which either have
a condition {\tt [SLASH NOSLASH]} on their {\tt N2} daughter, or no condition
on this daughter. The first and second formulae add conditions on the {\tt P2}
daughter in the output semantics.
\begin{ex}
\begin{verbatim}
PASS : VP --> W, N2. ==> VP[PAS] --> W, ( P2[by] ) :
   2 = [~SLASH], 5 = [~SLASH],
   (lambda (s)
      (lambda (y)
         ((lambda (x) ((lambda (2) (s x)) y)) 2))) : 
   2 = [~SLASH], 5 = [SLASH X2],
   (lambda (s)
      (lambda (y)
         (lambda (wh)
            ((lambda (x) ((lambda (2) (s x)) y)) (2 wh))))) : 
   2 = [~SLASH],
   (lambda (s)
      (lambda (y)
         ((lambda (2) (s (some (x) (entity x)))) y))).
\end{verbatim}
\end{ex}
When this metarule is applied to the ID rule
\begin{ex}
\begin{verbatim}
VP/TAKES_NP : VP --> H[SUBCAT NP], N2:
   (lambda (x) (1 x 2)).
\end{verbatim}
\end{ex}
the semantic formulae on the output rule containing the {\tt P2} (as the
second daughter) would be:
\begin{ex}
\begin{verbatim}
2 = [~SLASH], (lambda (y) (1 2 y))
2 = [SLASH X2], (lambda (y) (lambda (wh) (1 (2 wh) y)))
\end{verbatim}
\end{ex}
and the semantic formula on the rule without the {\tt P2} daughter would be
\begin{ex}
\begin{verbatim}
(lambda (y) (1 (some (x) (entity x)) y))
\end{verbatim}
\end{ex}


\section*{Metarule Semantic Operators}

In some cases, metarule semantic formulae may not be sufficiently powerful
to perform a desired transformation on ID rule semantics. Version 1.35
allows these transformations to be coded in Lisp, the compiled Lisp function
made the value of a symbol's property `semantic-operator' and the symbol's name
given as the metarule semantic formula. For example, the general case of
the transformation:
\begin{ex}
\begin{verbatim}
(lambda (x) (1 e x 2 (3 2))))
->
(lambda (x)
   (lambda (wh) (1 e x 2 (3 2 wh)))))
\end{verbatim}
\end{ex}
cannot be expressed as a lambda expression. However, it could be coded directly
in Lisp as
\begin{ex}
\begin{verbatim}
(defun slash-operator-1 (form idrule-binding-nos)
   ;; wrap (lambda (wh) ...) around body of form
   ...
   (let ((index-3 (nth 3 idrule-binding-nos)))
      ;; occurrences of index-3 in form refer to daughter number 3
      ...
      ;; insert wh as last argument of an application of daughter
      ;; number 3 semantics
      ...
      ))

(setf (get '|slash-operator-1| 'metarule-operator)
   (compile 'slash-operator-1))
\end{verbatim}
\end{ex}
and the metarule using this transformation as
\begin{ex}
\begin{verbatim}
SLASH : ... . ==> ... : slash-operator-1.
\end{verbatim}
\end{ex}
Note that the casing of the operator name must be the same where it is
set up and in the metarules in which it is referred to. The function must
be defined before the metarules are read in, and must be a compiled
function. It should take two arguments: the first the
ID rule semantic formula as a Lisp expression; the second a list
of integers mapping the position of the mother and daughters in the ID rule
to the numbers referring to the mother and daughters in the semantic formula
(the number referring to the mother is the zeroth element, the number referring
to the first daughter the first element, etc.). The function should return the
transformed formula.


\section*{Interface Changes}

In versions before 1.35, when running a core image of a previous GDE session
in which morphology system lexicons were loaded, the lexicons had to be 
reloaded manually before looking up a word otherwise Lisp would signal a
``file not open'' error. In version 1.35,
if the image is saved with the function `save-gde-image', the lexicon files
are automatically opened when needed, and the lexicons need not be 
reloaded.

Version 1.35 contains three new flags. They are:

\newcommand{\gdeflag}[2]{\item #1\\
#2}

\begin{list}{}
   {\leftmargin 1.6em
    \itemindent -\leftmargin
    \parsep 0pt plus 1pt}

\gdeflag{WIdth of terminal} {Can be set to the length (in characters) of
the current interaction window to control the point at which the GDE breaks
lines on output (when running in a setup in which the GDE itself does not
query the window width). The value must be an integer greater than 40, and is
initially 78. The Lisp variable holding the value of this flag is
{\tt *terminal-page-width}.}

\gdeflag{WArning messages} {Controls whether warning messages are
printed. The initial value is ON. The Lisp variable holding the value
of this flag is {\tt *warning-messages}. In version 1.35, a metarule
changing the value of a feature when constructing the output
rule is regarded as a normal occurence and no warning is ever printed
when this happens. However, a warning is now output when viewing the
semantics of a set of parse trees if one or more of the trees has no
semantics.}

\gdeflag{TAgged words} {When set to ON, words containing an underscore
character (`{\tt \_}') have the characters before the underscore removed before
they are looked up in the GDE lexicon or morph system. This allows the type of
tag sequences usually produced by lexical taggers to be parsed via the GDE
with minimal editing. The initial value is OFF, and the flag's value is
held in the Lisp variable {\tt *tagged-words}.}
\end{list}

In 1.35, changing the value of the `MOrphology system' flag unloads the
currently loaded lexicons only if the new flag value is OFF, or if the value
is not the same as the path of the one that was loaded first.


\section*{Dumped Grammar Format}

The internal format in which object grammars are held has been changed,
and so therefore has the type of output produced by the ``dump'' command
with the ``unreadable'' option.

The output consists of, firstly a list of the names of all the features in
the grammar; secondly a one-dimensional array, each element being
a list of features; and finally the object grammar PS rules.
A rule is represented as a list, the mother
being the first element, followed by the daughter categories in order.
A category is a dotted pair whose head is a vector of feature values,
and whose tail is, for a mother, a string representing the name
of the rule, or, for a daughter, a symbol indicating whether the
category is repeated one or more times (`{\tt +}'), occurs
exactly once (`{\tt nil}'), or is a gap (`{\tt |null|}').
The zeroth element of each feature value vector is an integer, an
index into the feature array, thus specifying what the features are
which correspond to the values. A value may be either a symbol or
another vector of values.

The following might be the first part of a dumped grammar file:
{\small
\begin{ex}
\begin{verbatim}
(N V BAR SUBJ SUBCAT CONJ VFORM H T ...)

#((BAR CONJ T BEGAP SLASH WH UB EVER COORD ELLIP)
  (NOSLASH) 
  (N V BAR SUBJ CONJ VFORM BEGAP FIN PAST PRD AUX INV COMP SLASH PRO WH UB
   EVER COORD AGR UDC ELLIP)
  ...)

((#(2 - + |2| + NULL NOT - + @12 @13 @14 - NORM @19 @93 @39 @40 @41 - 
    #(11 + - |2| @20 @21 @22 @23 NOM) - -) "S1a")
   (#(4 + - |2| NULL - - - #(1 +) @20 @21 @22 @23 NOM @ @ @ - + @ @ - @ @
      @39 @40 @41 @ @ NO @ @ @))
   (#(12 - + |2| - NULL NOT - + @12 @13 @14 @ @19 @93 - 
      #(11 + - |2| @20 @21 @22 @23 NOM) -)))
\end{verbatim}
\end{ex}
}

The ''dwords'' command with the ``unreadable'' option produces output in a
similar form to that produced by the ``dump'' command. A sequence of word
definitions follows the initial list of features and feature array.
A word definition is a dotted pair whose head is a
list of categories, one for each sense of the word, and whose tail is
the word itself. A category has as its head a list of feature / value
pairs, and tail the symbol {\tt t}. Thus a file containing a
definition for the word {\tt kim} might look like
{\small
\begin{ex}
\begin{verbatim}
(N V BAR SUBJ SUBCAT CONJ VFORM H T ...)

#(...
  (N V BAR SUBCAT CONJ PRD NFORM PER PLU COUNT CASE PN PRO PROTYPE PART
   POSS ADV NUM COORD REFL ADDRESS DEMON)
   ...)

((#(36 + - |0| NULL NULL @13 NORM |3| - + @24 + - NONE - - - - @44 @45 -
    -)
   . T)
 |kim|)
\end{verbatim}
\end{ex}
}


\section*{Miscellaneous}

In versions before 1.35, when semantic representations contained re-entrancies,
beta-reduction and variable renaming sometimes gave incorrect results. These
bugs are fixed in 1.35.

When using the GDE with the full release 4 ANLT grammar on a UNIX machine,
16 MBytes of real memory seems to be the absolute minimum to avoid thrashing.
With some Lisp implementations, this configuration requires 24 MBytes. On Apple
Macintoshes running MacOS, and PCs running MS-Windows, with this grammar the GDE
needs to be allocated at least 10 MBytes of memory. 

\end{document}
